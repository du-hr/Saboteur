\documentclass[12pt,twoside,letterpaper]{article}
%NOTE: This report format is
\usepackage{lipsum}
\bibliographystyle{plain}
\newenvironment{longlisting}{\captionsetup{type=listing}}{}
\setlength{\parskip}{1em}
% include files that load packages and define macros
\input{includes} % various packages needed for maths etc.
\input{notation} % short-hand notation and macros


%%%%%%%%%%%%%%%%%%%%%%%%%%%%

\begin{document}
% front page
% Last modification: 2016-09-29 (Marc Deisenroth)
% Modification for UW: 2017-05-22 (jphickey)
% Modification for UW: 2017-10-30 (jphickey)
\begin{titlepage}

\newcommand{\HRule}{\rule{\linewidth}{0.5mm}} % Defines a new command for the horizontal lines, change thickness here
\newgeometry{top=0.8in,bottom=1.5in,right=1.0in,left=1.0in}


%----------------------------------------------------------------------------------------
%	LOGO SECTION
%----------------------------------------------------------------------------------------



\begin{center} % Center remainder of the page

%----------------------------------------------------------------------------------------
%	HEADING SECTIONS
%----------------------------------------------------------------------------------------

\hspace*{-0.30in}
\includegraphics[width = 18cm]{./figures/mcgill_socs.pdf}\\[1.0cm]
\textsc{\Large Faculty of Science}\\[1.2cm]
\textbf{\textsc{\huge COMP 424 - Artificial Intelligence}}\\[1.0cm]
% \textsc{\Large Department of Electrical and Computer Engineering}\\[1.0cm]

%----------------------------------------------------------------------------------------
%	TITLE SECTION
%----------------------------------------------------------------------------------------

\HRule \\[0.4cm]
{ \huge \bfseries Project Report: Saboteur}\\
\HRule \\[1.5cm]
\end{center}
%----------------------------------------------------------------------------------------
%	AUTHOR SECTION
%----------------------------------------------------------------------------------------

%\begin{minipage}{0.4\hsize}
\vspace{0.04cm}
\begin{flushleft} \large
\textit{Authors:}\\
\begin{itemize}
  \item Haoran Du (ID: 260776911)
  \item Cameron Cherif (ID: 260784819)
\end{itemize}
\end{flushleft}
\vspace{6.6cm}
\makeatletter
\begin{flushright} \large
\@date
\end{flushright}
\vfill % Fill the rest of the page with whitespace



\makeatother


\end{titlepage}

%%%%%%%%%%%%%%%%%%%%%%%%%%% table of content
%If a table of content is needed, simply uncomment the following lines
\tableofcontents
\newpage
\newpage

%%%%%%%%%%%%%%%%%%%%%%%%%%%% Main document
\section{Introduction}
The project is based on a on a specific version of a card game called Saboteur, which was originally introduced back in 2004. In this version, there are 5 types of cards available, namely Tile, Malus, Bonus, Destroy, and Map, on a  restricted board of size (15,15). The goal of the project is to implement a computer agent with artificial intelligence called \mintinline{java}{StudentPlayer} defined in \mintinline{java}{public class StudentPlayer extends SaboteurPlayer}. \\ The AI agent is supposed to beat another predefined agent whose move in the game is set to be random.
\par There are many constrains to our project, both from the scope of the project itself and several external factors. In the end, the result we obtained was satisfactory. We had a success rate of 33.33\% from our testing rounds of play and we have satisfied all the requirements from the project specification document.

\section{Technical Approach}
\subsection{Description}
Our strategy is a modified version of the local greedy search algorithm to head towards the gold nugget objectives, which is based on the knowledge of Hill Climbing and Local Optimization we learned in class. In short, our general strategy is to use the Map cards to find the location of the goal, drop cards that are deemed discardable along the way (to maximize the chances of having a favourable hand), use the Malus cards aggressively whenever appropriate, and build a path with the best Tile card in hand, determined by the Local Optimization search, towards the goal.
\par In addition, an idea of aggressive approach was also included.
\par critical region
1.2.3.4 list below with code short
\subsection{Motivation}
The motivation behind this approach is mainly due to the composition of the deck. There is a total of 56 cards distributed among the players, both starting with 7 cards. Thus, there are only 46 turns, or 23 rounds to discard cards that do not fit one's strategy. Furthermore, only two "Malus" cards are present in the deck, for twice as many "Bonus" cards, therefore increasing the chances of countering a "Malus", which is unfavourable for an aggressive strategy. However, 6 "Map" cards are in the deck, thus composing almost a ninth of the deck, favouring a "rushing" strategy. Three "Destroy" cards can also compose the deck, which allow the destruction of a tile placed on the board, favourable for path building when one of the 9 dead-end tiles available in the deck are placed. The remaining 28 cards are "tunnel tiles" permitting route building in various directions.
\subsection{Theoretical Basis}
\subsubsection{Local Optimization Search}
\subsubsection{Constraint Satisfaction Problem (CSP)}
\clearpage

\section{Summary of Result}
\subsection{Satisfaction of the Project Specification}
\subsection{Success Rate}
\clearpage

\section{Reflect upon the Approach}
\lipsum
\clearpage

\section{Future Improvements}
\lipsum
\vfill
\section*{Students' contributions}
Both of student partners worked together to understand the problem and write the code in this project.
\end{document}
%%% Local Variables:
%%% mode: latex
%%% TeX-master: t
%%% End:
