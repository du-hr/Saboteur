\documentclass[12pt,twoside,letterpaper]{article}
%NOTE: This report format is
\usepackage{lipsum}
\bibliographystyle{plain}
\newenvironment{longlisting}{\captionsetup{type=listing}}{}

% include files that load packages and define macros
\input{includes} % various packages needed for maths etc.
\input{notation} % short-hand notation and macros


%%%%%%%%%%%%%%%%%%%%%%%%%%%%

\begin{document}
% front page
% Last modification: 2016-09-29 (Marc Deisenroth)
% Modification for UW: 2017-05-22 (jphickey)
% Modification for UW: 2017-10-30 (jphickey)
\begin{titlepage}

\newcommand{\HRule}{\rule{\linewidth}{0.5mm}} % Defines a new command for the horizontal lines, change thickness here
\newgeometry{top=0.8in,bottom=1.5in,right=1.0in,left=1.0in}


%----------------------------------------------------------------------------------------
%	LOGO SECTION
%----------------------------------------------------------------------------------------



\begin{center} % Center remainder of the page

%----------------------------------------------------------------------------------------
%	HEADING SECTIONS
%----------------------------------------------------------------------------------------

\hspace*{-0.30in}
\includegraphics[width = 18cm]{./figures/mcgill_socs.pdf}\\[1.0cm]
\textsc{\Large Faculty of Science}\\[1.2cm]
\textbf{\textsc{\huge COMP 424 - Artificial Intelligence}}\\[1.0cm]
% \textsc{\Large Department of Electrical and Computer Engineering}\\[1.0cm]

%----------------------------------------------------------------------------------------
%	TITLE SECTION
%----------------------------------------------------------------------------------------

\HRule \\[0.4cm]
{ \huge \bfseries Project Report: Saboteur}\\
\HRule \\[1.5cm]
\end{center}
%----------------------------------------------------------------------------------------
%	AUTHOR SECTION
%----------------------------------------------------------------------------------------

%\begin{minipage}{0.4\hsize}
\vspace{0.04cm}
\begin{flushleft} \large
\textit{Authors:}\\
\begin{itemize}
  \item Haoran Du (ID: 260776911)
  \item Cameron Cherif (ID: 260784819)
\end{itemize}
\end{flushleft}
\vspace{6.6cm}
\makeatletter
\begin{flushright} \large
\@date
\end{flushright}
\vfill % Fill the rest of the page with whitespace



\makeatother


\end{titlepage}

%%%%%%%%%%%%%%%%%%%%%%%%%%% table of content
%If a table of content is needed, simply uncomment the following lines
\tableofcontents
\newpage
\newpage

%%%%%%%%%%%%%%%%%%%%%%%%%%%% Main document
\section{Introduction}
\lipsum
\clearpage

\section{Technical Approach}
\subsection{Description and Theoretical Basis}
\lipsum
\clearpage

\subsection{Code Implementation}
The VHDL description of the synchronous 8-bit counter can be seen in Listing \ref{8bitcounter}.
\begin{longlisting}
\begin{minted}
[escapeinside=||,
frame=lines,
framesep=2mm,
baselinestretch=1.2,
linenos,
bgcolor=LightGray]
{vhdl}
library IEEE;
use IEEE.STD_LOGIC_1164.ALL;
use IEEE.STD_LOGIC_UNSIGNED.ALL;
entity g03_lab1 is
    Port(       clk         :in std_logic; |\label{clk}|
                countbytwo  :in std_logic;
                rst         :in std_logic;
                enable      :in std_logic;
                output      :out std_logic_vector(7 downto 0));
end g03_lab1;

architecture lab1 of g03_lab1 is
signal count: std_logic_vector (7 downto 0);
    begin
    process(clk) |\label{sl}|
        begin
        if (rising_edge(clk)) then |\label{rising}|
            if (rst = '1') then |\label{reset}|
                count <= "00000000";
            elsif (enable = '1' ) then |\label{enable}|
                if (countbytwo = '0') then
                    if (count = "11111111") then |\label{over1}|
                        count <= "00000000";
                    else
                        count <= count + 1;
                    end if;
                else
                    if (count = "11111110") then |\label{over2}|
                        count <= "00000000";
                    else
                        count <= count +2;
                    end if;
                end if;
            end if;
        end if;
    end process;

output <= count;

end architecture;
\end{minted}
\caption{Synchronous Dual Rate Counter in VHDL}
\label{8bitcounter}
\end{longlisting}
\par The counting logic is in the architecture block is summarized as following:
\begin{enumerate}
  \item To make the counter reset synchronous, only one parameter 'clk' (decalred in Line \ref{clk}) is included in the sensitivity list of the process block in Line \ref{sl}.
  \item A conditional statement in Line \ref{rising} ensures that the counter logic only executes on the rising clock edge.
  \item The reset bit is checked in Line \ref{reset}. If it is high, the counter is reset to zero.
  \item The enable bit is verified in Line \ref{enable} before checking the counter step size.
  \item In Line \ref{over1} and \ref{over2}, we check if the the counter reaches the maximum value allowed in the respective rate. If that is the case, the counter is reset to zero.
  \item The output would always be even if the rate is 2-bit since the counter adds 2 to itself every time there is a rising-edge clock signal.
\end{enumerate}
\clearpage

\subsection{Justification and Motivation for the Approach}
\lipsum
\clearpage

\section{Summary of Result}
\subsection{Satisfaction of the Project Specification}
\subsection{Success Rate}
\clearpage

\section{Reflect upon the Approach}
\lipsum
\clearpage

\section{Future Improvements}
\lipsum
\vfill
\section*{Students' contributions}
Both of student partners worked together to understand the problem and write the code in this project.
\end{document}
%%% Local Variables:
%%% mode: latex
%%% TeX-master: t
%%% End:
